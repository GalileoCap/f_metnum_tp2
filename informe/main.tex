%&pdflatex
\documentclass[12pt]{article}
\usepackage[margin=1.0in]{geometry}

\usepackage{g-util}
\usepackage{caratula.met}
\usepackage{amsfonts, amsmath, amssymb, amsthm}
\usepackage{graphicx, float, hyperref, wrapfig, subcaption}
\graphicspath{ {./plots/} }
%\usepackage[xetex]{xcolor}

\usepackage{subfiles}

% \theoremstyle{plain}
\newtheorem{prop}{Proposición}
\newtheorem{lema}{Lema}
% \theoremstyle{remark}
\newtheorem{obs}{Observación}
% \theoremstyle{definition}
\newtheorem{defi}{Definición}

\setcounter{MaxMatrixCols}{12}

\renewcommand{\figurename}{Fig.}
\renewcommand{\listfigurename}{Figuras}
\renewcommand{\tablename}{Tabla}
\renewcommand{\contentsname}{Secciones}
\renewcommand{\refname}{Bibliografía}

\begin{document}

\titulo{TP2 - Reconocimiento de Dígitos}
%\subititulo{}

\fecha{\today}

\materia{Métodos Numéricos}
\grupo{Grupo 8}

\integrante{Cappella Lewi, F. Galileo}{653/20}{galileocapp@gmail.com}
\integrante{Anachure, Juan Pablo}{99/16}{janachure@gmail.com}
\integrante{La Tessa, Octavio}{477/16}{octalate@hotmail.com}

\maketitle
\tableofcontents

\pagebreak

\pagebreak
\appendix 

\renewcommand{\thesection}{\Roman{section}}

\subfile{sections/appendix}

\pagebreak
\listoffigures

\begin{thebibliography}{9}

\end{thebibliography}

\end{document}
